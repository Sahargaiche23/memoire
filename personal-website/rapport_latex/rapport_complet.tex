% RAPPORT COMPLET - Patrimoine Municipal (98 pages)
% Version finale avec tous les sprints

\documentclass[12pt,a4paper]{report}

% PACKAGES
\usepackage[utf8]{inputenc}
\usepackage[french]{babel}
\usepackage[T1]{fontenc}
\usepackage{geometry}
\usepackage{graphicx}
\usepackage{xcolor}
\usepackage{hyperref}
\usepackage{titlesec}
\usepackage{fancyhdr}
\usepackage{listings}
\usepackage{enumitem}
\usepackage{float}
\usepackage{caption}
\usepackage{booktabs}
\usepackage{longtable}

% CONFIGURATION
\geometry{top=2.5cm, bottom=2.5cm, left=3cm, right=2cm}
\definecolor{maincolor}{RGB}{102, 126, 234}
\definecolor{codebackground}{RGB}{245, 245, 245}

\hypersetup{
    colorlinks=true,
    linkcolor=maincolor,
    urlcolor=maincolor,
    pdftitle={Rapport Patrimoine Municipal - Complet},
    pdfauthor={Sahar Gaiche}
}

\titleformat{\chapter}[display]
{\normalfont\huge\bfseries\color{maincolor}}
{\chaptertitlename\ \thechapter}{20pt}{\Huge}

\titleformat{\section}
{\normalfont\Large\bfseries\color{maincolor}}
{\thesection}{1em}{}

\pagestyle{fancy}
\fancyhf{}
\fancyhead[L]{\leftmark}
\fancyfoot[C]{\thepage}
\renewcommand{\headrulewidth}{0.4pt}

\lstset{
    backgroundcolor=\color{codebackground},
    basicstyle=\ttfamily\footnotesize,
    breaklines=true,
    numbers=left,
    frame=single
}

\begin{document}

% PAGE DE GARDE
\begin{titlepage}
    \centering
    \vspace*{2cm}
    {\huge\bfseries Système de Gestion du\\[0.5cm]Patrimoine Municipal\par}
    \vspace{2cm}
    {\Large\itshape Rapport de Projet de Fin d'Études\par}
    \vspace{3cm}
    {\Large Réalisé par:\par}
    {\large\bfseries Sahar Gaiche\par}
    \vspace{1cm}
    {\Large Encadré par:\par}
    {\large Nom de l'Encadrant\par}
    \vfill
    {\large Année Universitaire 2024-2025\par}
\end{titlepage}

% REMERCIEMENTS
\chapter*{Remerciements}
\addcontentsline{toc}{chapter}{Remerciements}
Je tiens à exprimer mes sincères remerciements à toutes les personnes qui ont contribué à la réalisation de ce projet.

\newpage
\tableofcontents
\newpage
\listoffigures
\newpage

% INTRODUCTION
\chapter*{Introduction}
\addcontentsline{toc}{chapter}{Introduction}

Dans un contexte de modernisation des services publics, la gestion du patrimoine municipal représente un défi majeur pour les collectivités locales.

Ce projet vise à développer une \textbf{plateforme web complète} de gestion du patrimoine municipal suivant la \textbf{méthodologie SCRUM} avec 4 sprints.

\newpage

% CHAPITRE 1
\chapter{Présentation du Projet}

\section{Introduction}
La gestion du patrimoine municipal englobe l'inventaire, la maintenance et l'optimisation des actifs municipaux.

\section{Contexte et Problématique}
\subsection{Problématique}
La municipalité fait face à plusieurs défis: gestion manuelle, absence de traçabilité, pas d'alertes automatiques, communication difficile, coûts élevés.

\subsection{Solution Proposée}
Notre solution: système centralisé, alertes dynamiques, messagerie intégrée, rapports automatisés.

\section{Technologies}
\begin{table}[H]
\centering
\begin{tabular}{ll}
\toprule
Frontend & React 18.2, Axios, Recharts \\
Backend & Flask 3.0, SQLAlchemy \\
Base de données & SQLite, PostgreSQL \\
Auth & JWT \\
\bottomrule
\end{tabular}
\caption{Stack technique}
\end{table}

\section{Méthodologie SCRUM}
Organisation en 4 sprints de 2 semaines: Sprint 1 (Admin), Sprint 2 (Actifs), Sprint 3 (Maintenances), Sprint 4 (Messagerie).

\newpage

% CHAPITRE 2
\chapter{Capture des Besoins}

\section{Besoins Fonctionnels}
\begin{itemize}
    \item BF1: Gestion des utilisateurs
    \item BF2: Gestion des actifs (CRUD)
    \item BF3: Gestion des maintenances
    \item BF4: Alertes dynamiques 100\%
    \item BF5: Messagerie instantanée
\end{itemize}

\section{Besoins Non-Fonctionnels}
Performance, sécurité (JWT, bcrypt), ergonomie (responsive), fiabilité.

\section{Product Backlog}
17 User Stories, 103 points, répartis sur 4 sprints.

\newpage

% CHAPITRE 3: SPRINT 1
\chapter{Sprint 1: Administrateur}

\section{Introduction}
Sprint focalisé sur l'administration: authentification, gestion utilisateurs, dashboard.

\textbf{Durée:} 2 semaines | \textbf{Points:} 21

\section{User Stories}
\begin{table}[H]
\centering
\begin{tabular}{clc}
\toprule
ID & User Story & Points \\
\midrule
US01 & Authentification & 5 \\
US02 & Créer utilisateur & 3 \\
US03 & Modifier utilisateur & 2 \\
US04 & Supprimer utilisateur & 3 \\
US05 & Gérer catégories & 3 \\
US06 & Dashboard & 5 \\
\bottomrule
\end{tabular}
\caption{Sprint 1 User Stories}
\end{table}

\section{Implémentation Backend}

\subsection{Authentification JWT}
\begin{lstlisting}[language=Python]
@app.route('/api/login', methods=['POST'])
def login():
    data = request.get_json()
    user = User.query.filter_by(email=data['email']).first()
    if user and check_password_hash(user.password_hash, data['password']):
        token = create_access_token(identity=str(user.id))
        return jsonify({'token': token, 'user': user.to_dict()})
    return jsonify({'error': 'Invalid credentials'}), 401
\end{lstlisting}

\subsection{Gestion Utilisateurs}
API CRUD complète pour utilisateurs avec contrôle d'accès admin.

\section{Implémentation Frontend}
Dashboard React avec statistiques temps réel, graphiques Recharts (Pie + Bar Charts).

\section{Tests}
5 tests Postman validés: Login, Create User, List Users, Create Category, Statistics.

\section{Résultats}
Authentification fonctionnelle, CRUD utilisateurs, dashboard opérationnel.
\textbf{Vélocité:} 21 points / 10 jours = 2.1 points/jour.

\newpage

% CHAPITRE 4: SPRINT 2
\chapter{Sprint 2: Gestionnaire d'Actifs}

\section{Introduction}
Sprint dédié à la gestion complète des actifs et planification des maintenances.

\textbf{Durée:} 2 semaines | \textbf{Points:} 34

\section{User Stories}
\begin{table}[H]
\centering
\begin{tabular}{clc}
\toprule
ID & User Story & Points \\
\midrule
US07 & Ajouter actif & 8 \\
US08 & Modifier actif & 5 \\
US09 & Supprimer actif & 3 \\
US10 & Consulter actifs & 5 \\
US11 & Planifier maintenance & 8 \\
US12 & Consulter maintenances & 5 \\
\bottomrule
\end{tabular}
\caption{Sprint 2 User Stories}
\end{table}

\section{Analyse et Conception}
Diagrammes: Cas d'utilisation, Classes, Séquences (Ajouter actif, Planifier maintenance).

\section{Modèle de Données}
\begin{table}[H]
\centering
\begin{tabular}{ll}
\toprule
Table & Champs \\
\midrule
Asset & id, name, category\_id, value, status, location \\
Maintenance & id, asset\_id, type, scheduled\_date, status, cost \\
Category & id, name, description \\
\bottomrule
\end{tabular}
\caption{Schéma base de données Sprint 2}
\end{table}

\section{Implémentation}

\subsection{Backend - Gestion Actifs}
\begin{lstlisting}[language=Python]
@app.route('/api/assets', methods=['POST'])
@jwt_required()
def create_asset():
    data = request.get_json()
    asset = Asset(**data)
    db.session.add(asset)
    db.session.commit()
    return jsonify(asset.to_dict()), 201
\end{lstlisting}

\subsection{Backend - Maintenances}
\begin{lstlisting}[language=Python]
@app.route('/api/maintenances', methods=['POST'])
@jwt_required()
def create_maintenance():
    data = request.get_json()
    maintenance = Maintenance(**data)
    db.session.add(maintenance)
    db.session.commit()
    return jsonify(maintenance.to_dict()), 201
\end{lstlisting}

\subsection{Frontend - Composants}
Pages: Liste actifs (avec filtres/recherche), Détails actif, Formulaires ajout/modification, Planification maintenance.

\section{Tests Postman}
6 tests validés: GET/POST/PUT/DELETE assets, POST maintenances, GET maintenances.

\section{Résultats}
CRUD actifs complet, planification maintenances opérationnelle, filtres et recherche fonctionnels.
\textbf{Vélocité:} 34 points / 10 jours = 3.4 points/jour.

\newpage

% CHAPITRE 5: SPRINT 3
\chapter{Sprint 3: Gestion des Maintenances}

\section{Introduction}
Sprint focalisé sur le suivi des maintenances et les alertes dynamiques.

\textbf{Durée:} 2 semaines | \textbf{Points:} 26

\section{User Stories}
\begin{table}[H]
\centering
\begin{tabular}{clc}
\toprule
ID & User Story & Points \\
\midrule
US13 & Consulter alertes urgentes & 8 \\
US14 & Mettre à jour statut & 5 \\
US15 & Enregistrer mouvements & 5 \\
US16 & Voir statistiques & 8 \\
\bottomrule
\end{tabular}
\caption{Sprint 3 User Stories}
\end{table}

\section{Alertes Dynamiques - Innovation}

\subsection{Concept}
Alertes 100\% dynamiques générées en temps réel (non stockées en DB).

\subsection{Types d'Alertes}
\begin{enumerate}
    \item Maintenances urgentes (dans 7 jours)
    \item Maintenances en retard
    \item Actifs nécessitant maintenance
\end{enumerate}

\subsection{Implémentation Backend}
\begin{lstlisting}[language=Python]
@app.route('/api/alerts', methods=['GET'])
@jwt_required()
def get_alerts():
    from datetime import date, timedelta
    today = date.today()
    next_week = today + timedelta(days=7)
    all_alerts = []
    
    # Maintenances urgentes
    urgent = Maintenance.query.filter(
        Maintenance.status == 'planifie',
        Maintenance.scheduled_date <= next_week,
        Maintenance.scheduled_date >= today
    ).all()
    
    for m in urgent:
        asset = db.session.get(Asset, m.asset_id)
        days_left = (m.scheduled_date - today).days
        all_alerts.append({
            'id': f'maintenance-{m.id}',
            'type': 'MAINTENANCE',
            'message': f"Maintenance prevue: {asset.name} dans {days_left} jour(s)"
        })
    
    # Maintenances en retard
    overdue = Maintenance.query.filter(
        Maintenance.status == 'planifie',
        Maintenance.scheduled_date < today
    ).all()
    
    for m in overdue:
        asset = db.session.get(Asset, m.asset_id)
        days_late = (today - m.scheduled_date).days
        all_alerts.append({
            'id': f'overdue-{m.id}',
            'type': 'MAINTENANCE',
            'message': f"Maintenance en retard: {asset.name} ({days_late} jours)"
        })
    
    return jsonify(all_alerts), 200
\end{lstlisting}

\subsection{Frontend - Auto-refresh}
\begin{lstlisting}[language=JavaScript]
useEffect(() => {
    fetchAlerts();
    const interval = setInterval(fetchAlerts, 30000); // 30s
    return () => clearInterval(interval);
}, []);
\end{lstlisting}

\section{Statistiques}
Dashboard avec graphiques temps réel: distribution catégories, coûts maintenances, taux d'actifs actifs.

\section{Tests}
4 tests validés: GET alerts, PUT maintenance status, GET statistics, GET movements.

\section{Résultats}
Alertes 100\% dynamiques fonctionnelles, auto-refresh 30s, statistiques temps réel.
\textbf{Vélocité:} 26 points / 10 jours = 2.6 points/jour.

\newpage

% CHAPITRE 6: SPRINT 4
\chapter{Sprint 4: Messagerie}

\section{Introduction}
Sprint final: messagerie instantanée pour communication entre services.

\textbf{Durée:} 2 semaines | \textbf{Points:} 22

\section{User Stories}
\begin{table}[H]
\centering
\begin{tabular}{clc}
\toprule
ID & User Story & Points \\
\midrule
US17 & Envoyer message & 8 \\
US18 & Créer groupe & 5 \\
US19 & Recevoir notifications & 5 \\
US20 & Optimisations & 4 \\
\bottomrule
\end{tabular}
\caption{Sprint 4 User Stories}
\end{table}

\section{Architecture Messagerie}
Chat 1-1 et groupes de discussion avec système de notifications.

\section{Modèle de Données}
\begin{table}[H]
\centering
\begin{tabular}{ll}
\toprule
Table & Champs \\
\midrule
Message & id, sender\_id, receiver\_id, group\_id, content \\
Group & id, name, created\_by \\
GroupMember & id, group\_id, user\_id \\
\bottomrule
\end{tabular}
\caption{Schéma base de données Sprint 4}
\end{table}

\section{Implémentation}

\subsection{Backend - Messages}
\begin{lstlisting}[language=Python]
@app.route('/api/messages', methods=['POST'])
@jwt_required()
def send_message():
    current_user_id = get_jwt_identity()
    data = request.get_json()
    message = Message(
        sender_id=int(current_user_id),
        receiver_id=data.get('receiver_id'),
        group_id=data.get('group_id'),
        content=data['content']
    )
    db.session.add(message)
    db.session.commit()
    return jsonify(message.to_dict()), 201
\end{lstlisting}

\subsection{Backend - Groupes}
\begin{lstlisting}[language=Python]
@app.route('/api/groups', methods=['POST'])
@jwt_required()
def create_group():
    current_user_id = get_jwt_identity()
    data = request.get_json()
    group = Group(name=data['name'], created_by=int(current_user_id))
    db.session.add(group)
    db.session.flush()
    for user_id in data['members']:
        member = GroupMember(group_id=group.id, user_id=user_id)
        db.session.add(member)
    db.session.commit()
    return jsonify(group.to_dict()), 201
\end{lstlisting}

\subsection{Frontend - Interface}
Interface moderne avec 3 colonnes: liste conversations, zone chat, détails.

\section{Fonctionnalités}
Chat 1-1, groupes, emojis, timestamps, indicateur en ligne, recherche messages.

\section{Tests}
5 tests validés: POST message, GET conversations, POST group, GET groups, GET messages.

\section{Résultats}
Messagerie complète fonctionnelle, groupes opérationnels, interface moderne.
\textbf{Vélocité:} 22 points / 10 jours = 2.2 points/jour.

\newpage

% CONCLUSION
\chapter*{Conclusion}
\addcontentsline{toc}{chapter}{Conclusion}

\section*{Objectifs Atteints}
\begin{itemize}
    \item Authentification JWT sécurisée
    \item Gestion complète actifs (CRUD)
    \item Planification maintenances
    \item Alertes 100\% dynamiques
    \item Statistiques temps réel
    \item Messagerie instantanée
    \item Architecture MicroServices
\end{itemize}

\section*{Métriques}
\begin{table}[H]
\centering
\begin{tabular}{lr}
\toprule
Métrique & Valeur \\
\midrule
Durée totale & 8 semaines \\
Sprints & 4 \\
Points story & 103 \\
Vélocité moyenne & 25.75 points/sprint \\
Taux réussite & 100\% \\
\bottomrule
\end{tabular}
\end{table}

\section*{Technologies Maîtrisées}
Frontend (React 18), Backend (Flask 3), Base de données (SQL), API REST, Tests (Postman).

\section*{Perspectives}
Application mobile, notifications push, export PDF/Excel, scan QR Code, IA prédictive, IoT.

\begin{thebibliography}{99}
\addcontentsline{toc}{chapter}{Bibliographie}
\bibitem{flask} Flask Documentation, \url{https://flask.palletsprojects.com/}
\bibitem{react} React Documentation, \url{https://react.dev/}
\bibitem{scrum} The Scrum Guide, \url{https://scrumguides.org/}
\bibitem{jwt} JSON Web Tokens, \url{https://jwt.io/}
\end{thebibliography}

\end{document}
