% ============================================================
% RAPPORT DE PROJET - Patrimoine Municipal
% Version corrigée pour Overleaf
% ============================================================

\documentclass[12pt,a4paper]{report}

% ============================================================
% PACKAGES ESSENTIELS
% ============================================================
\usepackage[utf8]{inputenc}
\usepackage[french]{babel}
\usepackage[T1]{fontenc}
\usepackage{geometry}
\usepackage{graphicx}
\usepackage{xcolor}
\usepackage{hyperref}
\usepackage{titlesec}
\usepackage{fancyhdr}
\usepackage{listings}
\usepackage{enumitem}
\usepackage{float}
\usepackage{caption}
\usepackage{booktabs}
\usepackage{longtable}
\usepackage{multirow}
\usepackage{amsmath}

% ============================================================
% CONFIGURATION GÉOMÉTRIE
% ============================================================
\geometry{
    top=2.5cm,
    bottom=2.5cm,
    left=3cm,
    right=2cm
}

% ============================================================
% COULEURS
% ============================================================
\definecolor{maincolor}{RGB}{102, 126, 234}
\definecolor{codebackground}{RGB}{245, 245, 245}

% ============================================================
% HYPERLINKS
% ============================================================
\hypersetup{
    colorlinks=true,
    linkcolor=maincolor,
    filecolor=maincolor,
    urlcolor=maincolor,
    citecolor=maincolor,
    pdftitle={Rapport Patrimoine Municipal},
    pdfauthor={Sahar},
    bookmarks=true,
}

% ============================================================
% TITRES
% ============================================================
\titleformat{\chapter}[display]
{\normalfont\huge\bfseries\color{maincolor}}
{\chaptertitlename\ \thechapter}{20pt}{\Huge}

\titleformat{\section}
{\normalfont\Large\bfseries\color{maincolor}}
{\thesection}{1em}{}

\titleformat{\subsection}
{\normalfont\large\bfseries}
{\thesubsection}{1em}{}

% ============================================================
% EN-TÊTES ET PIEDS DE PAGE
% ============================================================
\pagestyle{fancy}
\fancyhf{}
\fancyhead[L]{\leftmark}
\fancyfoot[C]{\thepage}
\renewcommand{\headrulewidth}{0.4pt}
\renewcommand{\footrulewidth}{0pt}

% ============================================================
% CODE LISTINGS
% ============================================================
\lstset{
    backgroundcolor=\color{codebackground},
    basicstyle=\ttfamily\footnotesize,
    breaklines=true,
    captionpos=b,
    numbers=left,
    numberstyle=\tiny\color{gray},
    frame=single,
    tabsize=2,
    showstringspaces=false
}

% ============================================================
% DÉBUT DU DOCUMENT
% ============================================================
\begin{document}

% ============================================================
% PAGE DE GARDE
% ============================================================
\begin{titlepage}
    \centering
    \vspace*{2cm}
    
    {\huge\bfseries Système de Gestion du\\[0.5cm]Patrimoine Municipal\par}
    
    \vspace{2cm}
    
    {\Large\itshape Rapport de Projet de Fin d'Études\par}
    
    \vspace{3cm}
    
    {\Large Réalisé par:\par}
    {\large\bfseries Sahar Gaiche\par}
    
    \vspace{1cm}
    
    {\Large Encadré par:\par}
    {\large Nom de l'Encadrant\par}
    
    \vfill
    
    {\large Année Universitaire\par}
    {\large\bfseries 2024-2025\par}
    
    \vspace{1cm}
    
\end{titlepage}

% ============================================================
% REMERCIEMENTS
% ============================================================
\chapter*{Remerciements}
\addcontentsline{toc}{chapter}{Remerciements}

Je tiens à exprimer mes sincères remerciements à toutes les personnes qui ont contribué à la réalisation de ce projet.

Tout d'abord, je remercie mon encadrant pour ses conseils précieux, son soutien constant et sa disponibilité tout au long de ce projet.

Je remercie également l'équipe de la municipalité pour leur accueil et leur collaboration, notamment pour avoir partagé leur expertise et leurs besoins réels en matière de gestion du patrimoine.

Mes remerciements s'adressent aussi à ma famille pour leur soutien inconditionnel et leurs encouragements durant toute ma formation.

\newpage

% ============================================================
% TABLES
% ============================================================
\tableofcontents
\newpage

\listoffigures
\newpage

\listoftables
\newpage

% ============================================================
% INTRODUCTION
% ============================================================
\chapter*{Introduction}
\addcontentsline{toc}{chapter}{Introduction}

Dans un contexte de modernisation des services publics et de digitalisation des processus administratifs, la gestion du patrimoine municipal représente un défi majeur pour les collectivités locales.

Le présent projet vise à développer une \textbf{plateforme web complète} de gestion du patrimoine municipal, permettant aux différents acteurs de collaborer efficacement dans la gestion quotidienne des actifs, des maintenances et des alertes.

Ce rapport présente l'ensemble du processus de développement du système, en suivant la \textbf{méthodologie Agile SCRUM} avec une organisation en \textbf{4 sprints} de deux semaines chacun.

\section*{Objectifs du Projet}

Les objectifs principaux de ce projet sont:

\begin{itemize}
    \item Centraliser la gestion de tous les actifs municipaux
    \item Automatiser les processus de maintenance préventive et corrective
    \item Alerter automatiquement sur les maintenances urgentes ou en retard
    \item Faciliter la communication entre les services
    \item Générer des rapports et statistiques en temps réel
    \item Optimiser les coûts de maintenance
\end{itemize}

\section*{Structure du Rapport}

Ce rapport est organisé comme suit:

\begin{description}
    \item[Chapitre 1] Présentation du projet, objectifs, outils et méthodologie SCRUM
    \item[Chapitre 2] Capture des besoins fonctionnels et non-fonctionnels
    \item[Chapitre 3] Sprint 1 - Fonctionnalités d'administration
    \item[Chapitre 4] Sprint 2 - Gestion des actifs
    \item[Chapitre 5] Sprint 3 - Gestion des maintenances et alertes dynamiques
    \item[Chapitre 6] Sprint 4 - Messagerie instantanée
    \item[Conclusion] Résultats obtenus et perspectives
\end{description}

\newpage

% ============================================================
% CHAPITRE 1: PRÉSENTATION
% ============================================================
\chapter{Présentation du Projet}

\section{Introduction}

La gestion du patrimoine municipal englobe l'ensemble des activités liées à l'inventaire, la maintenance, le suivi et l'optimisation des actifs appartenant à une municipalité.

Ces actifs incluent:
\begin{itemize}
    \item \textbf{Biens immobiliers:} Bâtiments administratifs, écoles, bibliothèques
    \item \textbf{Véhicules:} Bus municipaux, véhicules de service
    \item \textbf{Équipements:} Mobilier urbain, matériel informatique
    \item \textbf{Infrastructures:} Routes, ponts, éclairage public
\end{itemize}

\section{Contexte et Problématique}

\subsection{Problématique Actuelle}

La municipalité fait face à plusieurs défis:

\begin{itemize}
    \item Gestion manuelle avec registres papier dispersés
    \item Absence de traçabilité des maintenances
    \item Pas de système d'alerte automatique
    \item Communication difficile entre services
    \item Rapports incomplets
    \item Coûts de maintenance élevés
\end{itemize}

\subsection{Solution Proposée}

Notre solution apporte:

\begin{itemize}
    \item Système centralisé avec base de données unique
    \item Traçabilité complète de chaque actif
    \item Alertes automatiques en temps réel
    \item Messagerie intégrée
    \item Rapports automatisés
    \item Optimisation des coûts
\end{itemize}

\section{Technologies et Outils}

\subsection{Stack Technique}

Le projet utilise une architecture \textbf{MicroServices} moderne.

\begin{table}[H]
\centering
\begin{tabular}{ll}
\toprule
\textbf{Composant} & \textbf{Technologies} \\
\midrule
Frontend & React 18.2, Axios, Recharts \\
Backend & Flask 3.0, SQLAlchemy 2.0 \\
Base de données & SQLite (Dev), PostgreSQL (Prod) \\
Authentification & JWT (JSON Web Tokens) \\
API & REST, JSON \\
\bottomrule
\end{tabular}
\caption{Stack technique du projet}
\end{table}

\subsection{Outils de Développement}

\begin{itemize}
    \item IDE: Visual Studio Code
    \item Versioning: Git + GitHub
    \item API Testing: Postman
    \item UML: PlantUML
    \item Documentation: Markdown, LaTeX
\end{itemize}

\section{Méthodologie SCRUM}

\subsection{Principes SCRUM}

SCRUM est un framework agile qui permet de développer des projets complexes de manière itérative et incrémentale.

Les principes clés:
\begin{enumerate}
    \item \textbf{Itératif:} Développement par cycles courts (sprints)
    \item \textbf{Incrémental:} Ajout progressif de fonctionnalités
    \item \textbf{Adaptatif:} Ajustement continu selon feedback
    \item \textbf{Collaboratif:} Équipe auto-organisée
    \item \textbf{Transparent:} Visibilité totale sur l'avancement
\end{enumerate}

\subsection{Organisation en 4 Sprints}

\begin{table}[H]
\centering
\begin{tabular}{clc}
\toprule
\textbf{Sprint} & \textbf{Focus} & \textbf{Points} \\
\midrule
Sprint 1 & Administrateur & 21 \\
Sprint 2 & Gestionnaire d'Actifs & 34 \\
Sprint 3 & Gestion Maintenances & 26 \\
Sprint 4 & Messagerie & 22 \\
\midrule
\textbf{Total} & \textbf{8 semaines} & \textbf{103} \\
\bottomrule
\end{tabular}
\caption{Planification des sprints}
\end{table}

\newpage

% ============================================================
% CHAPITRE 2: CAPTURE DES BESOINS
% ============================================================
\chapter{Capture des Besoins}

\section{Introduction}

La phase de capture des besoins permet d'identifier précisément les besoins fonctionnels, non-fonctionnels, les acteurs et les cas d'utilisation du système.

\section{Besoins Fonctionnels}

\subsection{BF1: Gestion des Utilisateurs}

\begin{itemize}
    \item Inscription et authentification sécurisée
    \item Gestion des profils utilisateurs
    \item Attribution des rôles (Admin, Gestionnaire, Technicien)
    \item Modification du mot de passe
\end{itemize}

\subsection{BF2: Gestion des Actifs}

\begin{itemize}
    \item Ajouter, modifier, supprimer un actif
    \item Consulter la liste des actifs
    \item Rechercher et filtrer les actifs
    \item Consulter l'historique complet
    \item Enregistrer les mouvements
\end{itemize}

\subsection{BF3: Gestion des Maintenances}

\begin{itemize}
    \item Planifier une maintenance (préventive ou corrective)
    \item Consulter les maintenances planifiées
    \item Modifier ou annuler une maintenance
    \item Marquer une maintenance comme terminée
    \item Consulter l'historique
    \item Enregistrer les coûts
\end{itemize}

\subsection{BF4: Système d'Alertes Dynamiques}

Le système génère automatiquement des alertes pour:

\begin{itemize}
    \item Maintenances urgentes (dans les 7 prochains jours)
    \item Maintenances en retard
    \item Actifs nécessitant une maintenance
\end{itemize}

Les alertes sont \textbf{100\% dynamiques}, basées sur les données en temps réel, et s'actualisent automatiquement toutes les 30 secondes.

\subsection{BF5: Messagerie}

\begin{itemize}
    \item Envoyer un message à un utilisateur
    \item Créer un groupe de discussion
    \item Consulter les conversations
    \item Rechercher des messages
    \item Recevoir des notifications
\end{itemize}

\section{Besoins Non-Fonctionnels}

\subsection{Performance}

\begin{itemize}
    \item Temps de chargement $\leq$ 2 secondes
    \item Temps de réponse API $\leq$ 500ms
    \item Support de 100 utilisateurs simultanés
\end{itemize}

\subsection{Sécurité}

\begin{itemize}
    \item Authentification JWT sécurisée
    \item Hachage des mots de passe (bcrypt)
    \item Protection contre injections SQL
    \item Contrôle d'accès basé sur les rôles (RBAC)
    \item HTTPS en production
\end{itemize}

\subsection{Ergonomie}

\begin{itemize}
    \item Interface intuitive et moderne
    \item Design responsive (mobile, tablet, desktop)
    \item Feedback visuel pour chaque action
    \item Messages d'erreur clairs
\end{itemize}

\section{Product Backlog}

Le Product Backlog complet contient 17 User Stories réparties sur 4 sprints pour un total de 103 points de story.

\begin{table}[H]
\centering
\begin{tabular}{clcc}
\toprule
\textbf{ID} & \textbf{User Story} & \textbf{Priorité} & \textbf{Points} \\
\midrule
US01 & S'authentifier & Haute & 5 \\
US02 & Gérer les utilisateurs & Haute & 8 \\
US03 & Gérer les catégories & Haute & 3 \\
US05 & Ajouter un actif & Haute & 8 \\
US09 & Planifier maintenance & Haute & 8 \\
US11 & Voir alertes urgentes & Haute & 8 \\
US15 & Envoyer messages & Moyenne & 8 \\
\bottomrule
\end{tabular}
\caption{Product Backlog (extrait)}
\end{table}

\newpage

% ============================================================
% CHAPITRE 3: SPRINT 1 - ADMINISTRATEUR
% ============================================================
\chapter{Sprint 1: Administrateur}

\section{Introduction}

Le Sprint 1 se concentre sur la mise en place des fonctionnalités d'administration du système:
\begin{itemize}
    \item Authentification et gestion de session
    \item Gestion des utilisateurs
    \item Gestion des catégories d'actifs
    \item Dashboard administrateur avec statistiques
\end{itemize}

\textbf{Durée:} 2 semaines (10 jours ouvrables)\\
\textbf{Points de Story:} 21 points\\
\textbf{Objectif:} Établir les fondations du système

\section{User Stories du Sprint 1}

\begin{table}[H]
\centering
\begin{tabular}{clc}
\toprule
\textbf{ID} & \textbf{User Story} & \textbf{Points} \\
\midrule
US01 & Authentification & 5 \\
US02 & Créer utilisateur & 3 \\
US03 & Modifier utilisateur & 2 \\
US04 & Supprimer utilisateur & 3 \\
US05 & Gérer catégories & 3 \\
US06 & Consulter dashboard & 5 \\
\midrule
\textbf{Total} & & \textbf{21} \\
\bottomrule
\end{tabular}
\caption{User Stories Sprint 1}
\end{table}

\section{Implémentation}

\subsection{Backend - Authentification}

L'authentification utilise JWT (JSON Web Tokens) avec Flask-JWT-Extended:

\begin{lstlisting}[language=Python, caption=API Login]
@app.route('/api/login', methods=['POST'])
def login():
    data = request.get_json()
    email = data.get('email')
    password = data.get('password')
    
    user = User.query.filter_by(email=email).first()
    
    if user and check_password_hash(user.password_hash, password):
        token = create_access_token(identity=str(user.id))
        return jsonify({'token': token, 'user': user.to_dict()})
    
    return jsonify({'error': 'Invalid credentials'}), 401
\end{lstlisting}

\subsection{Frontend - Dashboard}

Le dashboard affiche les statistiques en temps réel avec des graphiques interactifs (Recharts).

\section{Résultats}

Le Sprint 1 a été complété avec succès:
\begin{itemize}
    \item Authentification JWT fonctionnelle
    \item Gestion complète des utilisateurs (CRUD)
    \item Dashboard avec statistiques et graphiques
    \item Tous les tests validés avec Postman
\end{itemize}

\textbf{Vélocité:} 21 points en 10 jours = 2.1 points/jour

\newpage

% ============================================================
% CONCLUSION
% ============================================================
\chapter*{Conclusion}
\addcontentsline{toc}{chapter}{Conclusion}

Ce projet de développement d'un système de gestion du patrimoine municipal a permis d'atteindre tous les objectifs fixés en appliquant rigoureusement la méthodologie SCRUM.

\section*{Objectifs Atteints}

\begin{itemize}
    \item Authentification sécurisée JWT
    \item Gestion complète des actifs (CRUD)
    \item Planification des maintenances
    \item Alertes dynamiques 100\% temps réel
    \item Statistiques et graphiques interactifs
    \item Messagerie instantanée avec groupes
    \item Architecture MicroServices scalable
\end{itemize}

\section*{Technologies Maîtrisées}

\begin{itemize}
    \item \textbf{Frontend:} React 18, React Router, Recharts
    \item \textbf{Backend:} Flask 3, SQLAlchemy, JWT
    \item \textbf{Base de données:} SQL, ORM
    \item \textbf{API:} REST, JSON
\end{itemize}

\section*{Métriques du Projet}

\begin{table}[H]
\centering
\begin{tabular}{lr}
\toprule
\textbf{Métrique} & \textbf{Valeur} \\
\midrule
Durée totale & 8 semaines \\
Nombre de sprints & 4 \\
Points de story & 103 \\
Vélocité moyenne & 25.75 points/sprint \\
Taux de réussite & 100\% \\
\bottomrule
\end{tabular}
\caption{Métriques finales du projet}
\end{table}

\section*{Perspectives d'Évolution}

Plusieurs pistes d'amélioration peuvent être envisagées:

\begin{enumerate}
    \item Application mobile (React Native) pour accès terrain
    \item Notifications push pour alertes critiques
    \item Export PDF/Excel avancé des rapports
    \item Scan QR Code pour identification rapide des actifs
    \item Dashboard analytics avec prédictions IA
    \item Intégration IoT pour monitoring temps réel
\end{enumerate}

\section*{Mot de Fin}

Ce projet a été une expérience enrichissante qui m'a permis de mettre en pratique mes connaissances en développement web full-stack et en gestion de projet agile.

% ============================================================
% BIBLIOGRAPHIE
% ============================================================
\begin{thebibliography}{99}
\addcontentsline{toc}{chapter}{Bibliographie}

\bibitem{flask}
Flask Documentation, \url{https://flask.palletsprojects.com/}

\bibitem{react}
React Documentation, \url{https://react.dev/}

\bibitem{scrum}
The Scrum Guide, Ken Schwaber and Jeff Sutherland, \url{https://scrumguides.org/}

\bibitem{jwt}
JSON Web Tokens Introduction, \url{https://jwt.io/introduction}

\bibitem{restapi}
RESTful API Design Best Practices, \url{https://restfulapi.net/}

\bibitem{uml}
UML Documentation, Object Management Group, \url{https://www.uml.org/}

\end{thebibliography}

\end{document}
